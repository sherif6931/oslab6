\documentclass[12pt]{article}
\usepackage[utf8]{inputenc}

\usepackage{ragged2e, geometry}
\usepackage{newpxtext,newpxmath}
\usepackage{indentfirst, setspace, fourier-orns}
\usepackage{fancyhdr, listings, xcolor, hyperref}
\usepackage[symbol]{footmisc}

\renewcommand{\thefootnote}{$\dagger$}
\renewcommand{\thesection}{\Roman{section}} 
\pagestyle{fancy}
\fancyhf{}
\fancyhead[C]{— {\thepage} —}
\fancyhead[R]{}
\fancyhead[L]{}
\fancyfoot[C]{}
\renewcommand{\headrulewidth}{0pt}

\lstset{
    language=C,
    basicstyle=\ttfamily\small,
    keywordstyle=\color{blue},
    commentstyle=\color{green},
    stringstyle=\color{red},
    numbers=left,
    numberstyle=\tiny,
    stepnumber=1,
    numbersep=5pt,
    breaklines=true,
    frame=single
}

\newcommand{\frontitle}[1]{
\begin{center}
    {\Huge #1} \\
    \vspace{2cm} 
\end{center}
}

\newcommand{\fullentry}[2]{
\thispagestyle{empty}
{\begin{center}
\section*{\huge{#1}}
\\ \vphantom{} \\
{\huge \\ CSE233 \\ \vphantom{} \\ \Large Sherif M. Haredy (I.D.: 223107334; MAJOR: C.S.) \\ GALALA UNIVERISTY, EGYPT \\ \vphantom{} \\ Prof.\ Amr Hefny \\ \vphantom{} \\ — \\ \vphantom{} \\ \today}
\end{center}}
\\ \vphantom{} \\
}

\begin{document}
\thispagestyle{empty}
\frontitle{SUBMISSION\,.}
\fullentry{‘‘Lab Report \#6.’’}

\section{Implementation}

A struct \texttt{job} handles the process related data:

\begin{itemize}
    \item PID number
    \item wait status
    \item interrupt code
    \item exit code
\end{itemize}

The handler routine \texttt{sigchld\_handler} works by checking whether the current process either equals \texttt{child2}'s PID or not:\\

\begin{lstlisting}
if (proc.p != child2.p)
\end{lstlisting}
\\ \vspace{0.1cm}
If it does not, it updates wait status and interrupt code; if it does, then it calls \texttt{\_exit(0)}, which terminates the parent if an interrupt signal is made.\\

\noindent The signal is sent via \texttt{kill()} according to the process' exit code.\\

The main routine forks two processes, \texttt{child1} and \texttt{child2} (prototyped globally), and runs:\\

\begin{lstlisting}
while (1) { }
\end{lstlisting}

\section{SIG Reader}

\emph{This routine is entirely ancillary and was not required per spec sheet.}\\

\texttt{rint.c} implements a small pipe that allows reading each \texttt{job} from each process, which saves the interrupt code; this allows inspection of how each process was interrupted.\\

A subroutine forks a \texttt{\_read} process which calls \texttt{sig\_read()}, feeding the interrupt code for the given child process.

\section{Testing}

The \texttt{script/} folder contains a shell script which runs a series of checks that match the lab's spec sheet. A CI runner works through it in order to verify success.\\

The errors depend on the value of \texttt{process.interrupt} and whether:\\

\begin{lstlisting}
printf("Child1 PID: %d Parent PID: %d\n\n", getpid(), getppid());
\end{lstlisting}
\\ \vspace{0.1cm}
messages are successful and correct, in order and format.

\section*{References}

\begin{itemize}
    \item Source Code: \url{https://github.com/sherif6931/oslab6}
    \item Lab Spec: c.f.\ \texttt{README.md} in the repository
\end{itemize}

\noindent\emph{This \text{\LaTeX} doc’s text is a copy of the {\tt dev} to {\tt main} successful PR.}
\end{document}
